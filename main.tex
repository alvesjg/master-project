\documentclass[12pt]{article}
\usepackage{packages}
\usepackage{xcolor}
\renewcommand{\baselinestretch}{1.3}

\sloppy

\begin{document}
\begin{center}
  
{\Large {\bf Clustering with outliers and fairness} 

}

\vspace{0.2cm}
{\small 
{\bf Candidate:} João Guilherme Alves Santos \\
{\bf Advisor:} Cristina Gomes Fernandes
}

\vspace{3mm} 
{\footnotesize \emph{This research project is part of the MSc scholarship application of João Guilherme Alves Santos.}}

\vspace{5mm} 

\begin{abstract}
Clustering refers to grouping objects based on similarities and is a widely known unsupervised learning task used in decision-making algorithms that affect the lives of millions of people.
Given the potential impact of these algorithms, fairness constraints are added to these problems to avoid external biases and ensure fairness in decisions.
The goal of this project is to study two clustering problems under fairness constraints: the colorful $k$-center and colorful $k$-median problems. 
These are clustering problems that allow for outliers, and aim at preventing a solution to choose as outliers mostly people from specific groups. All of these problems are of great interest in practice, and are known to be $\NP$-hard, so there is special interest in the design of good approximation algorithms for them. 
The focus of this project is on the investigation of existing and new approximation algorithms and inapproximability results for this class of clustering problems. 
This line of research lies in both combinatorial optimization and operations research.
The candidate has an excellent background in computer science in general, and specifically on the theme of the project and the line of research. 
% , and mastering the techniques used in these algorithms will improve the candidate's ability to develop and analyze approximation algorithms.
\end{abstract}

\end{center}

\section{Introduction and research justification} % com síntese da bibliografia fundamental
% introduce clustering and fairness
Clustering is the process of partitioning a given set of objects into groups based on their similarities. 
This fundamental unsupervised learning task plays a crucial role in a wide range of machine learning applications that impact the lives of millions of people, including image analysis, social network analysis, and pattern recognition. 
The results of these clustering processes can influence critical decisions in areas such as healthcare, marketing, and education, where groupings may shape opportunities and outcomes.
Given the significant social impact, ensuring fairness in algorithms for these problems has become a vital priority. 
This is where the field of \emph{fair clustering} emerges as an important area of study.
A broad tutorial on this topic can be found in~\cite{FairClusteringTutorial}.

Two classical clustering problems play a central role in combinatorial optimization: the $k$-center and the $k$-median problems. 
Given a point set $D$ (whose elements are called clients), a set of possible locations $F$ (whose elements are called facilities), a distance function $d : (F \cup D) \times (F \cup D) \rightarrow \mathbb{R}_{\geq 0}$ such that $(F\cup D, d)$ forms a \href{https://en.wikipedia.org/wiki/Metric_space}{metric space} and an integer~$k$, both problems ask for a set $S \subseteq F$ of cardinality at most $k$ that minimizes a cost involving the distance of each client to its closest facility in $S$.
The goal in the $k$-center problem is to minimize the maximum distance between any client and its closest facility in $S$, while, in the $k$-median problem, the goal is to minimize the sum of the distances between each client and its closest facility in $S$.
In the $k$-center problem, it is traditional to assume that $F=D$, while the general case (where $F$ and~$D$ can be distinct) is the \emph{$k$-supplier problem}.

The requirement of providing service to all clients is some times unrealistic, as it might increase the cost substantially, compared to a solution that addresses well most of the clients. 
So, one of the well-studied and widely applied variants of these problems is the version where we do not require all clients to be served; in other words, \emph{outliers} are allowed. 
In these variants, known as the \emph{robust $k$-center} and \emph{robust $k$-median} problems~\cite{CharikarKMN2001}, the input is the same as in the standard version, with the addition of an integer $t$ representing the minimum number of clients that must be served. 
The goal is to output a set $S \subseteq F$ of cardinality at most $k$ and a set $C \subseteq D$ of covered clients with cardinality at least $t$ that minimize the same objective as in the standard versions, but restricted to the set~$C$.  The elements in $D \setminus C$ are the so called outliers.  A significant issue that may arise is that certain groups might be entirely classified as outliers, leading to undesirable solutions.

Driven by the need for fair solutions, Bandyapadhyay \emph{et al.}~\cite{BIPV2019} introduced the \emph{colorful $k$-center} problem, which is a generalization of the robust $k$-center that aims at avoiding such undesirable solutions, ensuring some kind of fairness. This formulation can also be applied to generalize the robust $k$-median problem. 
In these variants, in addition to the input provided in the standard version, a partition $\set{D_1,\ldots, D_c}$ of~$D$ into $c$ color classes is provided, along with a coverage requirement $t_i$ for each color class~$i$. 
The goal is to output a set $S \subseteq F$ of cardinality at most $k$ and a set $C \subseteq D$ such that $|C \cap D_i| \geq t_i$ for every color class $i$, minimizing the maximum distance between any client in $C$ and its closest facility in $S$ for the colorful $k$-center, and minimizing the sum of the distances between each client in $C$ and its closest facility in $S$ for the colorful $k$-median.
If there is only one color class, then the definition matches that of the robust version.
Using the disparate impact doctrine, as articulated by Feldman \emph{et al.}~\cite{FSMSV2015} following the \href{https://en.wikipedia.org/wiki/Griggs_v._Duke_Power_Co.}{\emph{Griggs v.\ Duke Power Co.}\ US Supreme Court case}, the color classes can be viewed as protected attributes, such as race and gender. 
With the colorful version of these problems, we ensure that no protected class are underrepresented in the solutions produced by decision-making algorithms.

% dificuldade dos problemas e approach
So all of these problems are of great interest in practice. However, they are all known to be $\NP$-hard, which means that there is no polynomial-time algorithms for them in general unless $P=\NP$.  One possible approach to tackle such hard problems is the use of approximation algorithms for them.  These are algorithms that produce solution that might not be of minimum cost, but that are proven to have a cost that is not so far from this minimum. 

A lot of work has been done with approximation algorithms for the standard version of the two clustering problems we are considering.
For the $k$-center problem, Gonzalez~\cite{G1985} and, independently, Hochbaum and Schmoys~\cite{HS1985} gave 2-approximation algorithms for the problem, while Hsu and Nemhauser proved that an approximation algorithm with a better ratio exists only if $P=\NP$.
For the $k$-median problem, the best approximation algorithm known is a very recent $(2 + \eps)$-approximation due to Cohen-Addad \emph{et al.}~\cite{CGLS2025}, while Jain \emph{et al.}~\cite{JMS2002} showed that it is $\NP$-hard to approximate the problem with a ratio better than $1.763$.

For the version of the $k$-center that allows outliers, that is, the robust $k$-center, 
Chakrabarty \emph{et al.}~\cite{CGK2020} designed a 2-approximation. 
Since this variant is a generalization of the $k$-center problem, unless $P = \NP$, there is no approximation algorithm with ratio better than~2.
For the $k$-median with outliers, the best known approximation algorithm is a $(6.994 + \eps)$-approximation designed by Gupta~\emph{et al.}~\cite{GMZ2021}.
Chen \emph{et al.}~\cite{CHXXZ2024} developed a fixed-parameter tractable algorithm that produces a solution of cost at most $3 + \eps$ times the minimum cost. Its running time is polynomial on the size of $D$ and $F$ but is exponential on $k$.

For the colorful $k$-center, there is an algorithm that nearly matches the tight approximation guarantee of 2. Jia \emph{et al.}~\cite{JSS2020} designed a 3-approximation algorithm for this problem, which takes polynomial time as long as the numbers of colors is constant. For the colorful $k$-median problem, to our knowledge, no approximation algorithm is known.

% interesse justificado
Due to the importance in ensuring fairness in the decision-making algorithms used nowadays, it is clear that developing good approximation algorithms for these problems has a big practical interest. 
Furthermore, there is a lot of theoretical interest as well, as the two problems in their standard versions are classical combinatorial optimization problems, with a vast literature on them. Adapting approximation algorithms for these versions to ensure fairness is a challenge from the research perspective that might require new techniques and more sophisticated algorithms, and end up leading to improvements for several other variants of clustering problems. 

The existing results in the literature show that these problems have recently received significant attention and have been published in prestigious conferences and journals. This attests to the quality of these works and also justifies their study.

\section{Candidate}

The candidate holds a Bachelor's Degree in Computer Science from the Institute of Mathematics and Statistics of the University of São Paulo (USP), Brazil.  As attested by his undergraduate academic records, his GPA is 9.2 out of 10.  

During his undergraduate studies, the candidate attended several courses in the area of Theoretical Computer Science, achieving excellent grades. Some of these courses include:
\begin{itemize} 
    \item MAC0325 Combinatorial Optimization – Grade: 10/10; 
    \item MAC0694 Combinatorics 1 – Grade: 9.9/10; 
    \item MAC0450 Approximation Algorithms – Grade: 9.4/10; 
    \item MAC0320 Introduction to Graph Theory – Grade: 9.2/10. 
\end{itemize}

The candidate also participated on the USP undergraduate research program from March 2023 to December 2024, under the supervision of Prof.\ Cristina G.\ Fernandes.  In particular, during the year of 2024, the candidate received a scholarship from FAPESP.
His undergraduate research project funded by FAPESP (grant 2023/16197-0) laid the background for his current Master's degree project. 
It included the study of approximation algorithms and hardness results for the standard versions of the $k$-center, 
the $k$-median, and also the facility location problem. 
The project was divided into two parts: in the first part, the candidate studied all the results for the three problems 
presented in the books  ``\emph{The Design of Approximation Algorithms}''~\cite{books/WS} and 
``\emph{Approximation Algorithms}''~\cite{books/Vazirani} and, in the second part, he studied the paper 
``\emph{Approximating $k$-Median via Pseudo-Approximation}''~\cite{LS2012}, which, at the time, 
was the latest major breakthrough in the approximation algorithms for the $k$-median problem. 
The work done within this undergraduate project was presented as a poster at the \emph{32º Simpósio Internacional de Iniciação Científica e Tecnológica} of the University of São Paulo.

Studying the approximation algorithms for the standard versions of these three problems is crucial to the present project, as most approaches for the fair versions rely on these algorithms. This ensures that the candidate is well-equipped to investigate these problems under additional constraints. So the courses taken and his studies within his undergraduate research project ensures that the candidate possesses the necessary qualifications to work on the present project.  

The candidate also participated on other academic activities such as the \emph{``6º Workshop Paulista em Otimização, Combinatória e Algoritmos''} and participated in the weekly seminars of the Combinatorics, Optimization and Theory of Computing group in the Department of Computer Science from the University of São Paulo, Brazil.
Since February 2025, he is enrolled in the Master's Program at the same department.
During the first year in the Master's Program, the students are expected to complete the course requirements. 
This semester, the candidate is enrolled in the following courses: 
\begin{itemize} 
  \item MAC6711 Topics in Analysis of Algorithms;
  \item MAC6908 Semidefinite Programming and Applications;
  \item MAC6957 Advanced Data Structures.
\end{itemize}
This will further complement his background in algorithms and combinatorial optimization, key areas for the development of the present project.  Also, in March 10 to 14, the candidate will take part on the second School of Combinatorics, in Rio de Janeiro.

% \newpage

\section{Objectives}

%The goal of this project is to investigate a range of techniques used in approximation algorithms for fair clustering problems. 
%These techniques encompass several advanced topics in combinatorial optimization, including various LP rounding approaches. 

The goal of this project is to study, from the perspective of approximation algorithms, variants of two of 
the most important clustering problems in combinatorial optimization, incorporating fairness constraints.
One of the objectives is to further strengthen the background of the candidate in the area of approximation algorithms
and to consolidate his knowledge on clustering, preparing him to be able to produce new results in these areas.  
For this, he must learn how to approach an open problem by trying to use techniques that were successful in solving 
similar problems, many times in a creative way, that involves tackling the problem from different perspectives.  
This process might lead also to the development of new techniques, and eventually to new results. 
Concretely, a major goal is to work towards designing the first approximation algorithm for the colorful $k$-median. 
With this in mind, the candidate will study the existing literature on the colorful $k$-center and on the robust $k$-median, 
investigating the possibility to design an approximation algorithm for the colorful $k$-median based 
on the techniques applied in these related problems.

Given the challenges inherent in developing and analyzing effective approximation algorithms for hard problems, 
it is essential for the candidate to gain exposure to methods that foster deep understanding of the problems addressed.
Considering the background of the candidate, we believe this project is well suited for his Master's dissertation
even if no new results are achieved. 
In the dissertation required by the Master's Program, the candidate will present both the studied results and any new findings.
The project aims at equipping the candidate with profound knowledge of approximation algorithms 
and clustering problems with fairness concerns, preparing him for future academic endeavors, including his planned pursuit of a PhD degree.

\section{Work plan and project schedule}

The Master's Program in which the candidate is enrolled requires 48 credits from courses, which correspond to six regular courses. 
In the current semester, as already mentioned, the candidate is taking three courses: Advanced Data Structures, Semidefinite Programming and Applications, and Topics in Analysis of Algorithm. 

In parallel with these courses, the candidate is studying the paper ``\emph{Fair Colorful $k$-Center}''~\cite{JSS2020} (JSS2020). 
Initially, this paper focuses on the two-color case of the colorful $k$-center. 
Then it extends the results to any constant number of colors. 
The authors use two linear programs to show that there exists a ``pseudo-solution'' of cost at most twice that of an optimal solution. 
A \emph{pseudo-solution} is a choice of at most $k + \ell$ facilities to be open, where $\ell$ is a constant. In their scheme, the constant $\ell$ depends only on the number $c$ of colors, and $\ell = 1$ in the case of two colors.
Their strategy proceeds by choosing, in a greedy way, $\ell$ of the $k+\ell$ open facilities to be closed, reviewing what clients to attend so that to keep fulfilling the color requirements and, in the process, degrading the approximation ratio from 2 to at most 3.
The material in this paper will be described in details in the candidate dissertation, so in the next months the candidate will already start writing about these results.
% By closing one of the open facilities in this pseudo-solution using a greedy approach, they can guarantee that one of the colors meets the minimum number of clients covered, although this does not satisfy the requirement for the other color. 
% They then design a method to smartly close a facility and reassign clients to the remaining clusters, ensuring that the requirements for both colors are met. 
% This approach yields a 3-approximation for the colorful $k$-center problem.

In March, the candidate will also participate in the second \emph{Brazilian School of Combinatorics}, held at IMPA in Rio de Janeiro. 
The event will feature eleven plenary talks, two mini-courses, and daily poster presentations. 
The candidate also intend to start studying the paper ``\emph{Constant Approximation for $k$-Median and $k$-Means with Outliers
via Iterative Rounding}''~\cite{KLS2018} (KLS2018). Our main idea to obtain an algorithm for the colorful $k$-median is to try to adapt the strategy presented in this paper.  Both this and the JSS2020 papers use some common strategies, so the study of both of these papers seems essential.  Besides, they present a solid and general framework to be used in clustering problems that is an asset for anyone that wishes to work with this line of problems. 

In the second semester, the candidate will take the three remaining required courses.
He intends to enroll in courses on Integer Programming, Computational Complexity and Software Engineering. 
He will also finish studying the paper KLS2018 and write the part of his dissertation about it.
Simultaneously, he will prepare for the qualification exam required by the Master's Program, 
and will also start preparing the project for a Research Internship Abroad (RIA). 
There are some researchers abroad (e.g., in the USA and Switzerland) who work intensively on clustering problems, 
and we intend to contact some of them to verify the possibility of the candidate making a scientific visit to 
their research group.

In the first months of 2026, in the beginning of third semester, the candidate will undertake the qualification exam. 
During this exam, the candidate must present a dissertation proposal to a judging committee composed of his advisor, 
who will serve as the president, and two additional members of the scientific community that work in related areas.
At the same time, the candidate and the advisor will further investigate the design of an approximation algorithm 
for the colorful $k$-median problem using the techniques learned from the two previously studied papers, 
and from others that turn out to seem relevant to the goals of the project.  For instance, it was announced 
very recently a $(2 + \eps)$-approximation for the classic $k$-median~\cite{CGLS2025} (CGLSS2025). 
This is a breakthrough, but the paper is not yet available. As soon as it is available, 
we should look at it, as it certainly will bring interesting new ideas useful for $k$-median variants.
Possibly we might decide on including material from this paper in the dissertation.

In the fourth semester, the candidate intends to participate in the RIA program.
He will also write papers if any new results are achieved, and he will follow with his studies and complete his dissertation.

Furthermore, the candidate must also participate in scientific events in the field, whether national or international, 
whenever possible. He should also attend and deliver seminars in the research group of Combinatorics, Optimization and Theory of Computing at the Department of Computer Science, which currently includes 14 professors, 
2 postdoctoral researchers, 8 doctoral students, and 7 master's students.

The schedule of the main activities planned for the four semester is:

\begin{center}
{\footnotesize
\begin{tabular}{|l|c|c|c|c|}
\hline
\textbf{Activity} & \textbf{1st Sem} & \textbf{2nd Sem} & \textbf{3rd Sem} & \textbf{4th Sem} \\
\hline
Take courses                   & \(\surd\) & \(\surd\) &           &           \\
Participate in weekly seminars & \(\surd\) & \(\surd\) & \(\surd\) & \(\surd\) \\
Study the material in JSS2020  & \(\surd\) &           &           &           \\
Write about JSS2020            & \(\surd\) & \(\surd\) &           &           \\
Study the material in KLS2018  & \(\surd\) & \(\surd\) &           &           \\
Write about KLS2018            &           & \(\surd\) & \(\surd\) &           \\
Read CGLSS2025 and other related papers &  & \(\surd\) & \(\surd\) & \(\surd\) \\
Qualification exam            &            &           & \(\surd\) &           \\
Dissertation writing          & \(\surd\)  & \(\surd\) & \(\surd\) & \(\surd\) \\
Dissertation defense          &            &           &           & \(\surd\) \\
\hline
\end{tabular}
}
\end{center}
% \caption{Proposed Schedule of Activities}
% \end{table}


\section{Methods and materials}
The materials used will be the papers mentioned above and, possibly, other important papers that can deepen the candidate's understanding of the problems studied.

The candidate will interact actively with the supervisor throughout the research process. 
One of the main aspects of this interaction will involve writing portions of the study and research materials. 
This practice serves several purposes, including assessing the candidate’s level of understanding and allowing the supervisor to provide timely feedback and guidance as needed. 
This iterative feedback process ensures that the candidate’s research approach remains on track and aligned with the project’s objectives.

In conclusion, this research methodology combines the acquisition of essential tools with extensive reading of relevant research papers. 
The candidate’s active interaction with the supervisor, including writing study and research materials, guarantees a focused and aligned research approach. 
By following this methodology, the project enables the exploration of advanced topics and the achievement of the project’s goals.

\section{Format and analysis of the results}

The candidate's performance will be evaluated primarily through his performance in the courses he takes. In addition, his participation in the group's seminars will also be assessed. 
Every topic studied will be written up in a text that will serve as a model for the dissertation, and this will also serve as an evaluation of the candidate's progress. 
If original and relevant results are obtained, they will be submitted to conferences in the field and to specialized journals. 
Since this is a master’s program, we do not consider this a requirement, but it may occur naturally. 
We believe it is important for the candidate to obtain a solid education that will prepare him well for a doctoral program; accordingly, much of his effort should be directed towards achieving that goal.
% \begin{enumerate}
% \itemsep0em 
% \item Estudo do problema $k$-centros: resultados de inaproximabilidade, algoritmo guloso e método do gargalo. Em curso.
% \item Estudo do problema de localização de instalações: método primal-dual, de arredondamento e inaproximabilidade.
% \item Estudo do problema de localização de instalações: busca local, método guloso e probabilístico.
% \item Escritas parciais do texto, referente a 1, 2 e 3.
% \item Estudo do problema $k$-mediana: busca local. 
% \item Preparação do relatório intermediário.
% \item Estudo do problema $k$-mediana: método primal-dual usando relaxação Lagrangeana, método de arredondamento 
% probabilístico e inaproximabilidade.
% \item Estudo de resultados mais recentes para o problema de localização de instalações e/ou $k$-mediana.
% \item Continuação da escrita do texto.
% \item Preparação do relatório final.
% \end{enumerate}

%\section{Material e métodos}

%\section{Forma e análise dos resultados}

\bibliographystyle{plain}
\bibliography{fairness}

\end{document}


